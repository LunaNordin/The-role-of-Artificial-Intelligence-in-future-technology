%% paper_template.tex is a modification of:
%% bare_conf.tex 
%% V1.2
%% 2002/11/18
%% by Michael Shell
%% mshell@ece.gatech.edu
%% 
%% This is a skeleton file demonstrating the use of IEEEtran.cls 
%% (requires IEEEtran.cls version 1.6b or later) with an IEEE conference paper.
%% 
%% Support sites:
%% http://www.ieee.org
%% and/or
%% http://www.ctan.org/tex-archive/macros/latex/contrib/supported/IEEEtran/ 
%%
%% This code is offered as-is - no warranty - user assumes all risk.
%% Free to use, distribute and modify.

% *** Authors should verify (and, if needed, correct) their LaTeX system  ***
% *** with the testflow diagnostic prior to trusting their LaTeX platform ***
% *** with production work. IEEE's font choices can trigger bugs that do  ***
% *** not appear when using other class files.                            ***
% Testflow can be obtained at:
% http://www.ctan.org/tex-archive/macros/latex/contrib/supported/IEEEtran/testflow


% Note that the a4paper option is mainly intended so that authors in
% countries using A4 can easily print to A4 and see how their papers will
% look in print. Authors are encouraged to use U.S. letter paper when 
% submitting to IEEE. Use the testflow package mentioned above to verify
% correct handling of both paper sizes by the author's LaTeX system.
%
% Also note that the "draftcls" or "draftclsnofoot", not "draft", option
% should be used if it is desired that the figures are to be displayed in
% draft mode.
%
% This paper can be formatted using the % (instead of conference) mode.
%++++++++++++++++++++++++++++++++++++++++++++++++++++++
%\documentclass[conference]{IEEEims} % Modified for MTT-IMS
%\documentclass[conference]{IMSTemplate}
\documentclass[a4paper, conference]{IEEEtran}
%++++++++++++++++++++++++++++++++++++++++++++++++++++++
% If the IEEEtran.cls has not been installed into the LaTeX system files, 
% manually specify the path to it:
% \documentclass[conference]{../sty/IEEEtran} 


% some very useful LaTeX packages include:

%\usepackage{cite}      % Written by Donald Arseneau
                        % V1.6 and later of IEEEtran pre-defines the format
                        % of the cite.sty package \cite{} output to follow
                        % that of IEEE. Loading the cite package will
                        % result in citation numbers being automatically
                        % sorted and properly "ranged". i.e.,
                        % [1], [9], [2], [7], [5], [6]
                        % (without using cite.sty)
                        % will become:
                        % [1], [2], [5]--[7], [9] (using cite.sty)
                        % cite.sty's \cite will automatically add leading
                        % space, if needed. Use cite.sty's noadjust option
                        % (cite.sty V3.8 and later) if you want to turn this
                        % off. cite.sty is already installed on most LaTeX
                        % systems. The latest version can be obtained at:
                        % http://www.ctan.org/tex-archive/macros/latex/contrib/supported/cite/

%\usepackage{graphicx}  % Written by David Carlisle and Sebastian Rahtz
                        % Required if you want graphics, photos, etc.
                        % graphicx.sty is already installed on most LaTeX
                        % systems. The latest version and documentation can
                        % be obtained at:
                        % http://www.ctan.org/tex-archive/macros/latex/required/graphics/
                        % Another good source of documentation is "Using
                        % Imported Graphics in LaTeX2e" by Keith Reckdahl
                        % which can be found as esplatex.ps and epslatex.pdf
                        % at: http://www.ctan.org/tex-archive/info/
% NOTE: for dual use with latex and pdflatex, instead load graphicx like:
%\ifx\pdfoutput\undefined
%\usepackage{graphicx}
%\else
%\usepackage[pdftex]{graphicx}
%\fi
%+++++++++++++++++++++++++++++++++++++++++++
% Added to commands
\input epsf
\usepackage{graphicx}
\usepackage[
    backend=biber,
    style=numeric,
    sorting = none,
    natbib=true,
    url=false, 
    doi=false,
    eprint=false
]{biblatex}
\addbibresource{cites.bib}
\usepackage{hyperref}
\hypersetup{
    colorlinks=true,
    citecolor = blue,
    linkcolor=blue,
    filecolor=magenta,      
    urlcolor=blue,
    pdftitle={Overleaf Example},
    }
%+++++++++++++++++++++++++++++++++++++++++++
% However, be warned that pdflatex will require graphics to be in PDF
% (not EPS) format and will preclude the use of PostScript based LaTeX
% packages such as psfrag.sty and pstricks.sty. IEEE conferences typically
% allow PDF graphics (and hence pdfLaTeX). However, IEEE journals do not
% (yet) allow image formats other than EPS or TIFF. Therefore, authors of
% journal papers should use traditional LaTeX with EPS graphics.
%
% The path(s) to the graphics files can also be declared: e.g.,
% \graphicspath{{../eps/}{../ps/}}
% if the graphics files are not located in the same directory as the
% .tex file. This can be done in each branch of the conditional above
% (after graphicx is loaded) to handle the EPS and PDF cases separately.
% In this way, full path information will not have to be specified in
% each \includegraphics command.
%
% Note that, when switching from latex to pdflatex and vice-versa, the new
% compiler will have to be run twice to clear some warnings.


%\usepackage{psfrag}    % Written by Craig Barratt, Michael C. Grant,
                        % and David Carlisle
                        % This package allows you to substitute LaTeX
                        % commands for text in imported EPS graphic files.
                        % In this way, LaTeX symbols can be placed into
                        % graphics that have been generated by other
                        % applications. You must use latex->dvips->ps2pdf
                        % workflow (not direct pdf output from pdflatex) if
                        % you wish to use this capability because it works
                        % via some PostScript tricks. Alternatively, the
                        % graphics could be processed as separate files via
                        % psfrag and dvips, then converted to PDF for
                        % inclusion in the main file which uses pdflatex.
                        % Docs are in "The PSfrag System" by Michael C. Grant
                        % and David Carlisle. There is also some information 
                        % about using psfrag in "Using Imported Graphics in
                        % LaTeX2e" by Keith Reckdahl which documents the
                        % graphicx package (see above). The psfrag package
                        % and documentation can be obtained at:
                        % http://www.ctan.org/tex-archive/macros/latex/contrib/supported/psfrag/

%\usepackage{subfigure} % Written by Steven Douglas Cochran
                        % This package makes it easy to put subfigures
                        % in your figures. i.e., "figure 1a and 1b"
                        % Docs are in "Using Imported Graphics in LaTeX2e"
                        % by Keith Reckdahl which also documents the graphicx
                        % package (see above). subfigure.sty is already
                        % installed on most LaTeX systems. The latest version
                        % and documentation can be obtained at:
                        % http://www.ctan.org/tex-archive/macros/latex/contrib/supported/subfigure/

%\usepackage{url}       % Written by Donald Arseneau
                        % Provides better support for handling and breaking
                        % URLs. url.sty is already installed on most LaTeX
                        % systems. The latest version can be obtained at:
                        % http://www.ctan.org/tex-archive/macros/latex/contrib/other/misc/
                        % Read the url.sty source comments for usage information.

%\usepackage{stfloats}  % Written by Sigitas Tolusis
                        % Gives LaTeX2e the ability to do double column
                        % floats at the bottom of the page as well as the top.
                        % (e.g., "\begin{figure*}[!b]" is not normally
                        % possible in LaTeX2e). This is an invasive package
                        % which rewrites many portions of the LaTeX2e output
                        % routines. It may not work with other packages that
                        % modify the LaTeX2e output routine and/or with other
                        % versions of LaTeX. The latest version and
                        % documentation can be obtained at:
                        % http://www.ctan.org/tex-archive/macros/latex/contrib/supported/sttools/
                        % Documentation is contained in the stfloats.sty
                        % comments as well as in the presfull.pdf file.
                        % Do not use the stfloats baselinefloat ability as
                        % IEEE does not allow \baselineskip to stretch.
                        % Authors submitting work to the IEEE should note
                        % that IEEE rarely uses double column equations and
                        % that authors should try to avoid such use.
                        % Do not be tempted to use the cuted.sty or
                        % midfloat.sty package (by the same author) as IEEE
                        % does not format its papers in such ways.

%\usepackage{amsmath}   % From the American Mathematical Society
                        % A popular package that provides many helpful commands
                        % for dealing with mathematics. Note that the AMSmath
                        % package sets \interdisplaylinepenalty to 10000 thus
                        % preventing page breaks from occurring within multiline
                        % equations. Use:
%\interdisplaylinepenalty=2500
                        % after loading amsmath to restore such page breaks
                        % as IEEEtran.cls normally does. amsmath.sty is already
                        % installed on most LaTeX systems. The latest version
                        % and documentation can be obtained at:
                        % http://www.ctan.org/tex-archive/macros/latex/required/amslatex/math/



% Other popular packages for formatting tables and equations include:

%\usepackage{array}
% Frank Mittelbach's and David Carlisle's array.sty which improves the
% LaTeX2e array and tabular environments to provide better appearances and
% additional user controls. array.sty is already installed on most systems.
% The latest version and documentation can be obtained at:
% http://www.ctan.org/tex-archive/macros/latex/required/tools/

% Mark Wooding's extremely powerful MDW tools, especially mdwmath.sty and
% mdwtab.sty which are used to format equations and tables, respectively.
% The MDWtools set is already installed on most LaTeX systems. The lastest
% version and documentation is available at:
% http://www.ctan.org/tex-archive/macros/latex/contrib/supported/mdwtools/


% V1.6 of IEEEtran contains the IEEEeqnarray family of commands that can
% be used to generate multiline equations as well as matrices, tables, etc.


% Also of notable interest:

% Scott Pakin's eqparbox package for creating (automatically sized) equal
% width boxes. Available:
% http://www.ctan.org/tex-archive/macros/latex/contrib/supported/eqparbox/



% Notes on hyperref:
% IEEEtran.cls attempts to be compliant with the hyperref package, written
% by Heiko Oberdiek and Sebastian Rahtz, which provides hyperlinks within
% a document as well as an index for PDF files (produced via pdflatex).
% However, it is a tad difficult to properly interface LaTeX classes and
% packages with this (necessarily) complex and invasive package. It is
% recommended that hyperref not be used for work that is to be submitted
% to the IEEE. Users who wish to use hyperref *must* ensure that their
% hyperref version is 6.72u or later *and* IEEEtran.cls is version 1.6b 
% or later. The latest version of hyperref can be obtained at:
%
% http://www.ctan.org/tex-archive/macros/latex/contrib/supported/hyperref/
%
% Also, be aware that cite.sty (as of version 3.9, 11/2001) and hyperref.sty
% (as of version 6.72t, 2002/07/25) do not work optimally together.
% To mediate the differences between these two packages, IEEEtran.cls, as
% of v1.6b, predefines a command that fools hyperref into thinking that
% the natbib package is being used - causing it not to modify the existing
% citation commands, and allowing cite.sty to operate as normal. However,
% as a result, citation numbers will not be hyperlinked. Another side effect
% of this approach is that the natbib.sty package will not properly load
% under IEEEtran.cls. However, current versions of natbib are not capable
% of compressing and sorting citation numbers in IEEE's style - so this
% should not be an issue. If, for some strange reason, the user wants to
% load natbib.sty under IEEEtran.cls, the following code must be placed
% before natbib.sty can be loaded:
%
% \makeatletter
% \let\NAT@parse\undefined
% \makeatother
%
% Hyperref should be loaded differently depending on whether pdflatex
% or traditional latex is being used:
%
%\ifx\pdfoutput\undefined
%\usepackage[hypertex]{hyperref}
%\else
%\usepackage[pdftex,hypertexnames=false]{hyperref}
%\fi
%
% Pdflatex produces superior hyperref results and is the recommended
% compiler for such use.



% *** Do not adjust lengths that control margins, column widths, etc. ***
% *** Do not use packages that alter fonts (such as pslatex).         ***
% There should be no need to do such things with IEEEtran.cls V1.6 and later.


% correct bad hyphenation here
\hyphenation{op-tical net-works semi-conduc-tor IEEEtran}
\begin{document}

% paper title
%\title{Submission Format for IMS2014 (Title in 24-point Times font)}
% If the \LARGE is deleted, the title font defaults to  24-point.
% Actually, 
% the \LARGE sets the title at 17 pt, which is close enough to 18-point.
%+++++++++++++++++++++++++++++++++++++++++++
\title{\LARGE The role of Artificial Intelligence in future technology}
%+++++++++++++++++++++++++++++++++++++++++++
% author names and affiliations
% use a multiple column layout for up to three different
% affiliations
%+++++++++++++++++++++++++++++++++++++++++++
%\author{\authorblockN{J. Clerk Maxwell}
%\authorblockA{School of Electrical and\\Computer Engineering\\
%Somewhere Institute of Technology\\
%City, State 54321--0000\\
%Email: maxwell@curl.edu}
%\and
%\authorblockN{Michael Faraday}
%\authorblockA{(List authors on this line using 12 point Times font\\ - use a second line if necessary)\\
%Microwave Research\\
%City, State/Region, Mail/Zip Code, Country\\
%Email: homer@thesimpsons.com}
%\and
%\authorblockN{Andr\'e M. Amp\`ere \\ }
%\authorblockA{Starfleet Academy\\
%San Francisco, CA 96678-2391\\
%Telephone: (800) 555--1212\\
%Fax: (888) 555--1212}}

\author{\authorblockN{Frederik Schittny \\ \small Submitted as part of an application for the Computer Science (M.Sc.) program \\ at the Technical University of Munich}}


%+++++++++++++++++++++++++++++++++++++++++++++++++++

% avoiding spaces at the end of the author lines is not a problem with
% conference papers because we don't use \thanks or \IEEEmembership


% for over three affiliations, or if they all won't fit within the width
% of the page, use this alternative format:
% 
% Another example.
%\author{\authorblockN{Michael Shell\authorrefmark{1},
%Homer Simpson\authorrefmark{2},
%James Kirk\authorrefmark{3}, 
%Montgomery Scott\authorrefmark{3} and
%Eldon Tyrell\authorrefmark{4}}
%\authorblockA{\authorrefmark{1}School of Electrical and Computer Engineering\\
%Georgia Institute of Technology,
%Atlanta, Georgia 30332--0250\\ Email: mshell@ece.gatech.edu}
%\authorblockA{\authorrefmark{2}Twentieth Century Fox, Springfield, USA\\
%Email: homer@thesimpsons.com}
%\authorblockA{\authorrefmark{3}Starfleet Academy, San Francisco, California 96678-2391\\
%Telephone: (800) 555--1212, Fax: (888) 555--1212}
%\authorblockA{\authorrefmark{4}Tyrell Inc., 123 Replicant Street, Los Angeles, California 90210--4321}}



% use only for invited papers
%\specialpapernotice{(Invited Paper)}

% make the title area
\maketitle

% \begin{abstract}
% \end{abstract}
\IEEEoverridecommandlockouts
% \begin{keywords}
% \end{keywords}
% no keywords

% For peer-reviewed papers, you can put extra information on the cover
% page as needed:
% \begin{center} \bfseries EDICS Category: 3-BBND \end{center}
%
% for peer-review papers inserts a page break and creates the second title.
% Will be ignored for other modes.
\IEEEpeerreviewmaketitle



\section{Introduction}
Since Artificial Intelligence (AI) is most broadly defined as intelligence exerted by computers or other artificial machines \cite[p. 29]{aiModernApproach}, the classification of what is an AI system and what is not heavily depends on the underlying definition of intelligence \cite[p. 31]{aiModernApproach}. Because AI is a field of research in computer science rather than a specific technology \cite[p. 1]{aiStructuresStrategies}, it includes a variety of subfields like learning, reasoning, knowledge representation and data analysis \cite[pp. 20-30]{aiStructuresStrategies}.\\

The introduction of AI as a computer science research field took place in 1956 during the Dartmouth Summer Research Project on Artificial Intelligence \cite{dartmouthFiftyYears}. However, theoretical considerations of artificial and machine-based intelligence date back earlier, for example in philosophy \cite[pp. 5-6]{Flasiński2016} or science fiction \cite[pp. 29-40]{riseOfSelfReplicators}. Notable early forms of AI research predating the Dartmouth Research Project include Turing's considerations on machine-based intelligence \cite{Turing1950ComputingMA}, as well as McCulloch and Pitts introduction of artificial neurons as mathematical structures \cite{mcculloch43a}.\\

The development of AI over the following decades up to today was not characterized by continuous progress, but rather by alternating phases of so-called AI boom and AI winters \cite{AiWinterLessons, briefHistory}. The AI booms were phases of consecutive research breakthroughs like the 1960s, in which research focused on the first uses of neural networks and language processing \cite[pp. 65-69]{aiModernApproach} or the AI boom from 1980 to 1987 with the widespread use of expert systems \cite[pp. 71-74]{aiModernApproach}. More modern phases of AI boom include the years between 2000 and 2011 in which AI was widely incorporated into consumer applications by big technology companies \cite{voiceAssistants}, as well as the 2010s in which the main focus of AI development was on deep learning and commercial big data analysis \cite[pp. 77-79, p.5]{aiModernApproach, briefHistory}. The phases in between, often referred to as AI winters, were periods of reduced academic progress, frequently accompanied or initiated by lack of funding, insufficient computational power and a lack of public interest \cite{AiWinterLessons, aiModernApproach}. The exact classification varies, but the phases from 1974 to 1980 and 1987 to 2000 are often considered the two major AI winters \cite{Toosi_2021, aiModernApproach}.\\

Today, AI systems are used in various fields including finance \cite[p. 1845]{aiModernApproach}, medical applications \cite[p. 84]{aiModernApproach}, manufacturing \cite[p. 293]{aiStructuresStrategies}, gaming \cite[pp. 45-46]{aiStructuresStrategies}, logistics \cite[pp. 82-83]{aiModernApproach} and military applications \cite[pp. 86, 1031]{aiModernApproach}. Despite the widespread use of AI in consumer applications like text autocompletion and correction, voice assistants or search and recommendation algorithms and their importance in the everyday life of many people, these systems are often not recognized or perceived as AI applications \cite{studyAIusage}.\\

\section{Recent advancements}
The last boost of AI development was mainly driven by AI models based on the transformer architecture introduced in 2017 \cite{transformer}. Especially the introduction of GPT-3 by OpenAI in 2020 \cite{brown2020language} marked the beginning of a variety of different tools like chatbots \cite{googleGemini, openAiChatGPT}, AI-powered search engines \cite{perplexity} as well as tools generating audiovisual materials based on text prompts \cite{midjourney, dalle}. Transformer-based AI models were quickly incorporated into numerous consumer software products and saw rapid adoption by end users \cite[pp. 437-451]{AiIndexReport24}. This advancement marks the milestone of AI models being able to mass generate texts and audiovisual content, which is difficult to recognize as non-human generated by most people \cite{frank2023representative}. The extreme popularity of these tools makes this milestone comparable to other AI milestones with high media attention, like the chess AI Deep Blue beating world chess champion Garry Kasparov in 1997. In contrast to other consumer software products mentioned above, the new transformer-based models are highly recognized as AI technology by most consumers \cite[pp. 33-37]{littman2022gathering}.\\

\section{Current problems and near-future development}
The quickly advancing development of AI applications has highlighted several technical and societal problems. Solving these problems is the current effort of a variety of companies and research institutions. Since partial solutions and solution approaches are foreseeable or already under discussion, a look at current problems allows for a prediction of the near-future development of AI.\\

Technical problems include the reliability and explainability of AI models' outputs, as current models can experience so-called hallucinations \cite{huang2023survey} and struggle with logic and temporal coherence of information \cite{chen2023learning}. Solving this problem, ideally by making the models' processes comprehensible \cite[pp. 1312-1315]{aiModernApproach}, would allow for AI models to be used in more critical and high-risk scenarios. Another technical problem is the high energy demand of training processes, which poses a potential risk for efforts toward climate protection \cite{Dhar2020}. Smaller, more efficient models could not only alleviate the concerns about the climate impact of AI but would also allow on-device usage with less powerful hardware, often referred to as edge computing \cite{AiEdge}, making AI applications more accessible and easier to adapt in many situations.\\

Besides technical problems, there are also highly discussed societal concerns with current AI developments. This includes algorithmic biases discriminating against certain groups, mostly caused by biased training data \cite[pp. 1812-1818]{aiModernApproach}. Other concerns are copyright \cite{genAiCopyright} and privacy issues \cite[pp. 1807-1812]{aiModernApproach} caused by the mass data collection necessary for the training of large AI models, as well as misinformation issues \cite[p. 55]{littman2022gathering} caused by the new ability of AI models to produce realistic media content. While these problems do not directly impede the further adoption and development of current AI technology, they undermine the trust and reputation of AI as a technology in the view of many users. Therefore, specialized regulation and law enforcement as well as industry standards are necessary to prevent a potential future backlash against AI.\\

\section{From multimodal AI to AGI}
The long-term development goal stated by many companies and researchers is Artificial General Intelligence (AGI) \cite{openAiAgi, googleDeepMindAgi}. The definition is once again dependent on the conception of the term intelligence but is generally understood to describe a system possessing human-like intelligence rather than just imitating it \cite[pp. 88-89]{aiModernApproach}.\\

One approach to AGI is training increasingly large and capable multimodal models \cite{Fei2022}. These can not only process and output one form of data like text but several different ones, including audiovisual information and the ability to perform actions in digital or even real-world environments \cite{baltrušaitis2017multimodal}. While multimodality is a recent concept and can be designated a near-future development, there are not only discussions about whether current models show first features of AGI \cite[pp. 92-95]{bubeck2023sparks} but also doubt from experts whether these models understand their outputs in a human-like way \cite{mahowald2024dissociating}.\\

If AGI can match or surpass human intelligence, this new technology will present both new benefits and challenges. AGI could prove to be more reliable, versatile and capable than current AI models and could potentially take over many tasks performed by humans today, yielding improved productivity and efficiency in many areas. On the other hand, it would drastically amplify existing concerns like job displacement \cite[p. 1821-1826]{aiModernApproach} and the potential for weaponization \cite[1803-1807]{aiModernApproach} and could pose an existential threat to humanity \cite{Turchin2020-TURCOG-2}. These issues become especially concerning in the case of self-improving systems \cite[pp. 1832-1833]{aiModernApproach}.\\

Estimations of when AGI will be achieved vary widely \cite{scienceAgiDiscussion, zhang2022forecasting, Fjelland2020}. This is partly due to the vague definition of AGI and the lack of expressive tests. Early tests for identifying intelligent machines like the Turing test were arguably passed by programs not falling under the AGI definition \cite[p. 1796]{aiModernApproach}. The original approach of modelling human neurons has not proven effective in producing intelligence yet but is further pursued through research on whole-brain simulation \cite{Stiefel2019}. However, this and other approaches are still constrained by technical limits like computational power and availability of training data \cite[pp. 681-703]{aiStructuresStrategies}. Besides this, the inadequate understanding of intelligence itself makes developing approaches to AGI difficult \cite[pp. 24-29]{littman2022gathering}.\\

\section{Conclusion}
The near-future development of AI can be predicted with relative certainty based on research and first approaches to current problems. These point towards AI being further adopted into everyday life and taking over more tasks as availability and reliability improve. This progress could be slowed down either by missing regulations and unsolved societal problems impacting the reputation and acceptance of AI or by consequent overregulation.\\

The long-term goal of AGI is already specified however, a clear definition, working approaches and standardized testing methods do not yet exist. Up until now, the development of AI was characterized by alternating phases of AI boom and AI winters, and it is unlikely the current boom phase will continue indefinitely.\\

\printbibliography

\end{document}